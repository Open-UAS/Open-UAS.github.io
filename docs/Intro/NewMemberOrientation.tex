\documentclass{article}
\usepackage[utf8]{inputenc}
\usepackage{enumitem,amssymb}
\usepackage{hyperref}
\usepackage{natbib}
\usepackage{graphicx}
\newlist{todolist}{itemize}{2}
\setlist[todolist]{label=$\square$}

\title{OpenUAS New Member Orientation}
\author{}
\date{May 2020}

\begin{document}

\maketitle

\section{Introduction}
Please read this document over if you are new to the OpenUAS team. It contains important checklists, resources, and advice. If you have questions about anything in this document, reach out to the Project Leader or your respective Team Leader - contact information is listed below.

\begin{center}
\begin{tabular}{|c|c|c| }
\hline
\textbf{Role} & \textbf{Name} & \textbf{Email} \\
\hline
Project Leader & Chris Johannsen & chrisj17@iastate.edu \\ 
\hline
Electrical/Software Team Leader & John Levandowski & johnl@iastate.edu \\  
\hline
Structures Team Leader & Stephanie Jou & ssjou@iastate.edu \\
\hline
Flight Test Team Leader & John Edgren & edgren@iastate.edu \\
\hline
\end{tabular}
\end{center}


\section{First Steps - Checklist}
After officially becoming a member of the team, you will be added to our communication channels and documentation repositories.\\

\noindent \textbf{Please verify that you are added to...}

\begin{todolist}
	\item the Github repositiory
	\item the Google Drive folder
	\item the emailing list
	\item the main Groupme chat
	\item your team Groupme chat
\end{todolist}

\noindent Reach out to the Project Leader if you do not have access to any of the above items. 


\section{Navigating Documentation Sources}
Once you are added to the Google Drive folder and the Github repository, you have access to all of the team's documentation over the past 3+ years. This can be daunting at first, so please read the following sections if you are confused about where to start. 

\subsection{Github}
The OpenUAS team keeps much of our formal documentation in Github. Github is a tool that hosts git, a file versioning system. For more information on this, this website does a good job explaining Github and git: \url{https://blog.devmountain.com/git-vs-github-whats-the-difference/}. \\

\noindent After you have made a Github account and are added to the team's repository, start to take a look at our documentation. One good place to start is with the final reports created each semester by the team. These reports provide a variety of information including an overview of the project and teams, the status of the project, and the progress made by each sub-team for that semester. It also contains valuable lessons learned which may help you as you begin work on the project. The path to these reports is: \url{OpenUAS/Documentation/Final_Reports}.\\

\noindent Another good place to explore in the Github repository are the progress reports made by each team member (under the ProgressReports directory). These reports are great insight on what a member of the OpenUAS team works on on a week-to-week basis. You will also have to create your own progress reports as soon as you start receiving credit for the research, so these are helpful to look at to know what is expected. For information on creating these items, see Section 4 below. 

\subsection{Google Drive}
The OpenUAS Google Drive folder contains more of the team's informal documentation and work. In this folder, you can find everything from performance calculations spreadsheets to possible design configurations. There isn't one area that you should look at right away, but as you have time, explore the directories within this drive folder. You will learn a lot about what the team has worked on. 

\section{Personal Documentation}
Every student on the team is required by Dr. Rozier to create personal documentation during their time on the team. This includes weekly reports as well as a semester plan. Each is described in detail below. There are templates for these reports found in the Github repository under the ProgressReports directory. Create your own directory (folder) within the ProgressReports directory with your initials as the name. Then, copy both the weekly reports template and the semester plan template into your folder and fill in the necessary information. These documents are created in markdown, a fairly simple markup language. If you are having issues understanding this format, this resource might be useful for you: \url{https://daringfireball.net/projects/markdown/syntax}. It is critical that you complete this documentation by the end of the semester as this is what Dr. Rozier uses to give you a grade for the research credit. 

\subsection{Weekly Reports}
Weekly reports are exactly what they sound like: a written update on what you did during the week. This includes your objectives and deliverables from the previous week and if you completed them. You are also required to describe any road blocks you faced during the week and describe in more detail your overall accomplishments for the week. Finally, you create objectvies and deliverables for the next week. 

\subsection{Semester Plan}
At the beginning of each semester, you are required to create a semester plan. This semester plan contains your goals, deliverables, and plan to mitigate risks during the semester. There is also a table-calendar to fill out. It is often necessary to update your semester plan multiple times during the semester as new deliverables or tasks arise. 

\section{Advice}

\begin{enumerate}
	\item{Ask questions constantly - no question is dumb. Everyone on the team is willing to help you!}
	\item{Look into the Git and LaTeX resources early. It is up to you to make sure you understand these tools.}
	\item{Keep up-to-date on your weekly reports. It can be easy to get behind on this documentation, but it is a pain to make up as it is often difficult to 		remember what you worked on weeks ago.}
	\item{Find what you are passion about on the team. If you joined the team in one area but find out your interests are elsewhere on the project, talk to your Team or Project Leader about eventually switching areas.}
	\item{Ask Dr. Rozier for career advice. She is very knowledgeable and is always willing to help.}
\end{enumerate}

\end{document}
