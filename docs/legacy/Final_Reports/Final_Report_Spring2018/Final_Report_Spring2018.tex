\documentclass{article}
\usepackage[utf8]{inputenc}
\usepackage{lstautogobble}
\usepackage[export]{adjustbox}
\usepackage{graphicx}
\usepackage{changepage}
\usepackage{listings}
\usepackage{amsthm}
\usepackage{subcaption}
\usepackage{amssymb}
\usepackage{titlesec}
\usepackage{hyperref}

%Command to change name of table of contents
\renewcommand*\contentsname{Table of Contents}

%Command to start sections on new pages
\newcommand{\sectionbreak}{\clearpage}

%Create a new "Unlabeled section" that will be added to toc but not printed
\newcommand{\unlabeledsection}[1]{%
 \clearpage
  \par\refstepcounter{section}% Increase section counter
  \sectionmark{#1}% Add section mark (header)
  \addcontentsline{toc}{section}{\protect\numberline{\thesection}#1}% Add section to ToC
  }
  
\emergencystretch=1em

\title{OpenUAS:\\Second Semester Progress Report }
\author{ }
\begin{document}


%%TITLE PAGE%%
\maketitle

\newpage

%%TEAM PAGE%%
\begin{center}
\Large \textbf{The OpenUAS Team}

\vspace{1cm}

\large{
Abigail Gries\footnote[1]{ISU Department of Aerospace Engineering}\\ Logan Gross\footnotemark[1]\\Madison Harrington\footnotemark[1]\\ Chris Johannsen\footnotemark[1]\\ Catherine Sener\footnotemark[1]\\ Josh Wallin\footnote[2]{ISU Department of Electrical and Computer Engineering}
}\par

\end{center}

\newpage

%%TABLE OF CONTENTS%%

\tableofcontents

%%PROJECT OVERVIEW%%
\section{Project Overview}

\subsection{Purpose}
Currently, there are no open-source unmanned aerial systems (UAS) which are fixed-wing and conceptually accessible available to the general public. There are some similar UAS which are available, but they must be purchased and are not open-source. OpenUAS is producing an open-source, commercial off-the-shelf (COTS) UAS that can be used for recreational and research purposes, and it will only consist of components available to the general public, including open-source software.

\subsection{Scope}
In order to develop an open-source, COTS UAS for recreational and research purposes that is free and available to the general public, a list of objectives, deliverables, and constraints were identified at the conception of the project. The following section will provide an overview of these lists.

\subsection{Objectives}
\begin{enumerate}
\item Create an open-source, COTS UAS for short recreational and research flights
\item Provide full documentation of the conception, design, and testing of all systems
\item The UAS should not require a runway for takeoff or landing
\item The piloting of the UAS shall be accessible to hobbyists
\item The UAS shall be made of ``affordable" components
\item UAS components should be reconfigurable and support additional components
\end{enumerate}

\subsection{Deliverables}
\begin{enumerate}
\item A functioning design and prototype of a UAS
\item A ground launch system for takeoff
\item Relevant tools for piloting the UAS from the ground
\item Extensive documentation of the development process
\item Extensive documentation on proper use and safety
\end{enumerate}

\subsection{Constraints}
\begin{enumerate}
\item COTS components
\item Affordable components
\item Easily duplicated components (e.g. all 3D printed parts can be reasonably produced by hobbyists)
\item All components should be reasonably safe (e.g. battery)
\item Design and testing completed within 1 academic year
\end{enumerate}

%%BACKGROUND%%
\section{Background}

\subsection{Related Work}

%%PROJECT DOCUMENTATION%%
\section{Project Documentation}

Varying forms of documentation were developed in order to assist with tracking the vision, objectives, high-level project plan, meeting documents to capture insight and actions needed, and so forth. The documentation is intended to not only support the systems engineering aspect of the project, but also serve an educational purpose by assisting future users who would like insight into what efforts and decisions were made to bring OpenUAS from a concept to a reality.

\begin{enumerate}
\item \textbf{Tollgate} \\Tollgates show what team member has primary responsibility for completing certain tasks, how team members / subsystems are cross-functional and share responsibility with completion of tasks, and also track what deliverables are associated with each objective. The Tollgates were set up to indicate the completion of major milestones, or ?phases?. Project Kick-Off has been completed, and Conceptual Design Review is currently being focused on as the semester reaches a close. Later, the Preliminary Design Review and Systems Design Review phases will be completed.
Tollgates become very helpful as a design begins to develop and ?take shape?, as it becomes a high-level review of what work has been accomplished. Each Tollgate has a list of ?Deliverables?--objectives that would show the project is ready to move to the next phase once all are completed. In order to show the deliverable is completed, ?objective evidence? is provided. (Example: ?Initiate Requirements? is a deliverable, and a requirements document would be the objective evidence.) The team must be in consensus that each deliverable has been met appropriately and that the project is on track to meet its objectives and purpose.
\item \textbf{Running Action Item List} \\The Running Action Item List (RAIL) is to be updated regularly with tasks as they come up. This is attached to the overall Tollgate document. Action items can be assigned to individuals or to the team, with an expected completion date.
The RAIL is not to be considered the same as ?Deliverables? and ?Objective Evidence? from the Tollgate document, although it should be noted that the objective evidence is often worked on as an action item. Many times, action items are assigned after team meetings to help ensure progress continues to be made.
\item \textbf{Project Charter}\\ This is a ?one-pager? intended to help provide a broad overview of the project. The Project Charter lays out the team, the scope of the project, the purpose, and so forth. Additionally, the project objectives, deliverables, and constraints that were mentioned earlier in this paper come from the Project Charter as it maintains the vision and intent of the OpenUAS project.
\item \textbf{Requirements}\\ The requirements document is currently a ?living? document that is still in progress. It includes the overall system requirements, as well as subsystem requirements. The requirements document helps guide the team in design decisions, but also has undergone multiple reviews, including one major peer review, to ensure that they are reasonable, feasible, and assist in meeting the overall project objectives. They are being updated to reflect the ?Easy Approach to Requirements Syntax? (EARS) format, which involves providing precise language and rationale. The revisions are maintained on GitHub.
\item \textbf{Weekly Meeting Agenda \& Minutes}\\Each week, Meeting Minutes are held to capture what discussion and decisions occurred. The tasks of leading the meeting and taking meeting minutes are rotated through the team. The one leading the meeting will set the agenda for that meeting?s primary discussion points, and the meeting minutes are taken in ?real-time? on the shared Google Drive. Afterward, the meeting minutes are then cleaned up and formatted into the meeting minute document kept on GitHub.
\item \textbf{Project Plan}\\ The Project Plan is a long-term, high-level calendar broken down by month and then broken down weekly with goals and deliverables. This document is maintained on GitHub. Every few weeks the team will make a plan for the next few months.
\item \textbf{Requirements}\\ The requirements document is an extensive, volatile artifact that has been developed since the beginning of the project. Originally, our primary requirements were high-level, and directly traceable to the overall project goals. Now, as we begin to make decisions about what our design will look like, the requirements have evolved to become more detailed. For the sake of making requirements easier to read and understand, we chose to use the EARS \cite{Terzakis2013} requirements syntax.
\end{enumerate}

\section{Related Work}
\noindent Currently, there are very few comparable fixed-wing UAS. The United States uses UAS such as the RQ-14A Dragon Eye and RQ-11B Raven in its military. Although these UAS are similar in size and weight to the OpenUAS team's target design, the technology and capabilities of these systems are much more advanced, and as such, the budget well exceeds the team?s overall budget.\\

\noindent The University of Virginia created the Razor, a small fixed-wing UAS for the Department of Defense. This UAS has a flying wing design and utilizes an Android phone as the main processor. The Razor is of similar size, weight, and performance of the team?s target design. One main difference in this system is that it is entirely 3D printed. The team plans on utilizing 3D printing, but not to the extent of the Razor design.\\

\noindent The Albatross is a commercial UAS produced by Applied Aeronautics. Although this aircraft is slightly larger than the team's target design, its performance and low-cost are comparable to the team's goals. This UAS is described in more detail later in the paper, as the team purchased and is beginning to study this design. \\


%%LAB SETUP AND ORGANIZATION%%
\section{Lab Set-Up \& Organization}
\noindent The lab provided this year began with no equipment. Throughout the year, orders had been placed for tools and other items needed in order to ensure a work space that has all tools needed for successful progression of this project. Because we are working within a large organization, special documentation and communication must be done in order to acquire the parts needed for the lab. This includes confirmation of successful retrieval of parts and checking if damage was done to them. If damage is found, proceeding with the proper return process and notifications so everyone knows how parts are moving about. If an item is not the item we purchased (ex a different sized cartridge for a label printer) we send it back to the individual keeping track of our orders and have them re-order the right equipment. \\

\noindent In order to organize and track many of the lab's tools, multiple drawers were labeled with the equipment found inside of them, and multiple tool kits and tool packages were ordered. This included everything from a standard tool box to a roll up precision tweezer kit. The tool kits have slots and divots to help with finding whether or not we have all of our tools in the correct location, maximizing our ability to stay on top of out tool management. 

\noindent In order to work with simulations, run flight software, and test certain electronic components, the team needed computers. Dr. Rozier, the Principal Investigator of the project, ordered and set up four computers using Linux operating systems. These computer would then be used by all teams using the lab. Since then, the team has converted one computer to be used mostly for the UAS project, 3D printing, and battery charging, while using another for communication with members remotely. Another computer is restricted due to projects being done on it, and the fourth is open for all team. The 3D printing computer will be moved once proper cables are put in place for internet connection, and will remain separated from the other three in order to better streamline movement within the lab. It will also sit directly across from the sink in order to move 3D printed parts over for cleaning and processing. \\

\subsection{3D Printer - LulzBot Taz 6}
\noindent The team has used the Lulzbot Taz 6 printer throughout the year. As no member of the team has used a 3D printer extensively before, it has taken awhile to become familiar with the printing process. The printer has had some issues throughout the year, and the team has slowly learned best practices and how to troubleshoot certain issues that may arise. Catherine and Abby created a running document in the git repo that contains relevant information to the printer. This file is in the documenation folder and explains everything from how to change filament to how to fix an uneven z-axis. The team plans on adding more information to this document as more experience with the printer is acquired. \\
\noindent Although the team has yet to print any structural components yet, the LulzBot has been very useful with the construction of the Albatross and with the wing design. The team printed servo holders for the Albatross and a model of the airfoil shape that will be used on the OpenUAS.\\

%%INSPIRATION: THE ALBATROSS%%
\section{Primary Inspiration: The Albatross UAV}
\noindent The team decided to purchase the Albatross, a low-cost, entry-level drone produced by Applied Aeronautics \cite{AlbatrossFAQ}, in order to have a model for testing ideas for the OpenUAS. Specifically, the team wanted to observe the flight characteristics such as power, controllability, maneuverability, speed, and airfoil characteristics. The team also wanted to order this model to look at the material of the structure, the inside configuration, the weight, and how easy it was to assemble. \\

\noindent After purchasing, the Albatross parts were delayed in shipping. The team received all parts before the end of the Fall 2017 semester, but did not start construction until the Spring 2018 semester. Delays in this construction were due to communication issues with Applied Aeronautics, insufficient parts, and the need for improvisition. The obstacles faced during the construction of the Albatross are in greater detail in the lessons learned section of this document. The team looks forward to getting the Albatross UAV into the air by the Summer of Fall 2018.\\

\noindent Unfortunately that goal will not be achieved quite that soon. The project took much longer than we anticipated due to the structual and electronic components having a long list of additional parts and changes to be made from the original parts given to us. 

\subsection{Construction and Progress}
\noindent Constructing the Albatross has taken the majority of the Spring 2018 semester. Some of the interior has been lined with velcro for easier attachment for the electronics. The holes for the LIDAR, air speed sensor, motor, and motor screws were drilled following the instruction from the Applied Aeronautic's construction manual with some improvisation. The servos all have 3-D printed holders and velcro attachments for easy placement and removal. The propellor pieces did not match what Applied Aeronautics suggested for assembly so the team drilled a larger hole in the propellor and secured it with some of the pieces in the kit. The team had to buy different screws than those that Applied Aeronautics supplied for the wheels and motor. \\
  
\noindent The team had to order more and longer wires to actually connect the avionics together. The team soldered the base of the iron-bird for the Albatross after getting these wires.Instead of soldering the servo motor wires together though, we are using a twist-on wire connector with an addition 10 AWG wire as an output to the connection so that the servos can gain power from the battery while giving us more flexibility with our wire management and reducing bad connection issues if rough treatment of the Albatross occurs (ex crashing). We have completed the base Albatross iron-bird for connecting the engine and the servos, and the system now has power. The team also set up the pixhawk so that the sensors communicate with the computer. The next step is getting the iron-bird to connect with the pixhawk and securing it inside the Albatross for flight testing. \\
  
\subsection{Looking forward}
\noindent The team is going to complete the electronic assembly and iron-bird testing. The Albatross itself required, and still requires, multiple holes and points drilled and cut into it for various systems and sensors. The biggest part we have left are the servo holes. The Albatross comes with multiple points started, but not drilled through. Parts that are now able to be used are the motor, the lidar, the front wheel, and the button. Once this is finished and works in the lab, the team will set up the Albatross to start flight testing. The greatest take away from the Albatross, thus far, has been what not to do, and our team is quite ready to respond to these obstacles by making the OpenUAS much more builder friendly. More detailed comments are in the Lessons Learned section of this report.




%%SYSTEM ARCHITECTURE AND PROGRESS%%
\section{OpenUAS System Architecture \& Progress}
\noindent The OpenUAS team divided the entire system into six subsystems: Structures, Software, Controls, Electronics, Propulsion, and Ground System. The entire system completed 3 requirement revisions with added rationale. The OpenUAS will be modeled with ideas taken from the Albatross, created with support from the Lulzbot 3-D Printer, and utilize the best, most accessible most accessible characteristics of fixed-wing UAS. More details on the OpenUAS system is discussed below in the corresponding subsystems. \\

\begin{figure}
\includegraphics{UAS_Subsystems.pdf}
\caption{Initial version of system architecture}
\centering
\end{figure}

\subsection{Structures}
\noindent The first step in the development of the structures subsystem was the creation of the structures' requirements. The requirement process is discussed in greater detail in the requirements section of this document. For the structures' requirements specifically, there are currently 25 requirements that have gone under numerous revisions and one formal peer review. There is also rationale written for each requirement, and the requirements are parametrized as needed. \\

\noindent After the preliminary round of structures' requirements were drafted, the next big step was determining a preliminary estimate of the total weight of the UAS. This step was key in the progress of the project because an estimate of the weight was needed to make successive design decisions from both a structural perspective and electronics perspective. This estimate of weight was determined using several different approaches. First, an estimation of the weight of the electronics and battery was determined from specifications of commonly used components on UAVs. Then, several combinations of wingspan, wing planform area, and the resulting aspect ratio were identified. Additionally, the maximum W/S ratio of the Albatross UAV was calculated. Using these combinations coupled with a W/S ratio slightly less than the Albatross, a maximum weight estimation was determined for each combination of wingspan, wing area, and aspect ratio.\\

\noindent The maximum weight calculated for each scenario was then compared to the sum of the estimated structural, electronic, and battery weights. The most favorable combination was selected. This configuration of the UAS requires a 5 ft wingspan and 2 ft wing platform area, creating an aspect ratio of 12.5, and allowing a maximum takeoff weight of 6 lbs. The estimations for weights were as follows: battery weight of 1.5 lbs, electronics weight of 1.5 lbs, and structural weight of 2.05 lbs, for a total weight of 5.05 lbs. This allows for almost 1 lb of "wiggle" room in the weight for estimation error or additional components.\\

\noindent The team decided that for the first design of the UAS, the wings would be constructed of foam. This decision was made based on the ease of working with foam, how light it is, and the availability of foam to an average hobbyist. However, an airfoil as thin we need made out of foam would not be strong enough to hold the fuselage and electronics up. So it was determined that the wings must be reinforced with some stronger material. Carbon fiber rods were sugested as they are thin and light. The foam considered for the wings were expanded polystyrene (EPS), expanded polypropylene (EPP), and extruded polystyrene (XPS). After an attempt at constructing the wings using a hot wire and EPP foam (see Lesson’s Learned section), it was decided that the first prototype of the wings will be built from XPS foam by the Wist Lab in Howe. This will allow the team to test the wing design out, while also avoiding using the hotwire at this time. Once the team decides on a final wing design after testing, the wings will be constructed using the hot wire or similar technique in order for the UAS to be accessible to hobbyists and students.\\

\noindent After capturing the structures requirements, estimating the weight, and selecting the preliminary materials, the next big step in the design process was the selection of an airfoil. This was an imperative step in the overall design of the wing, as the performance of the UAS can be severely limited due to an inappropriate airfoil. As the UAS will be flying at low Reynold’s numbers and low velocities, three airfoils that had dominant airspeeds in the team’s target airspeed range were compared against one another. Through XFLR5 analysis and calculations of key airfoil characteristics, the three different airfoils were each assigned a weighted score. The airfoil with the highest score in the end was chosen as the team’s airfoil. The chosen airfoil is NACA 4512.\\

\noindent The team decided on wingspan, chord length, wing planform area, and aspect ratio requirements for the UAS wing design in the Fall 2017 semester. Using this information coupled with the selected airfoil, Catherine created the first CAD model of the wings. This model is also in the process of fluid-dynamic analysis through Star+CCM.\\

\noindent In order to confirm our calculations are correct, and to get exact values, the full UAS model should go through a fluid dynamic analysis software. As of now, the wings are in the process of being run through Star+CCM. All models have been created with the potential to go through Star+CCM as Star has specific requirments for CAD models.\\

\noindent Another major step in the design process was center of gravity calculations of the UAS. This was an important step in the design process as it allowed the team to select an appropriate location for the wing, calculate the vertical and horizontal tail areas, and is the beginning step in verifying that the UAS will be stable. To find the center of gravity, each component inside the fuselage, along with the fuselage itself, was weighed. Then, the distance from the nose of the fuselage to each components’ placement in the fuselage was measured. From that, the moment about the nose was calculated and the center of gravity was determined. Further iterations of this process will be completed as components are added, removed, or repositioned in the fuselage.\\

\noindent A consideration with the tail is the base design of a split empennage, such as that of the Albatross, or a traditional empennage. As of now our fuselage design puts the propeller at the front of the UAS so a split tail is not required. So our first prototype will have a traditional empennage design, but we plan on putting the propeller in the back for our next design, making a split tail required for aerodynamic purposes.\\

\subsection{Software}
\noindent The goals surrounding software for this year included selecting a suitable existing, open source flight computer software, develop a list of useful requirements, and test said requirements for realizability using existing model checking software. \\

\noindent Upon completing our research of open source flight software options, we chose PX4 (https://github.com/px4/Firmware/), a community-developed autopilot for UAS. PX4 contains not only the necessary components for manual, but also autonomous flight. As we had debated the level of autonomy for our final product, this degree of freedom will be useful. The open source nature of PX4 allows for it to be continuously developed by hobbyists and professionals in the field; the well-structured community ensures that PX4 becomes more and more powerful, with approval from a dedicated team required to make substantial changes. Additionally, the flight software can be edited and built in the lab, which should allow us to alter values and behaviors as needed to satisfy our overarching goals. While ArduPilot presented an alternative autopilot software, frequently used in the UAS domain, we felt that PX4 was more well supported, documented, and developed for our chosen flight computer, the Pixhawk. \\

\noindent The Pixhawk is an onboard flight computer that includes a barometer, three degree of freedom (DOF) gyroscope, three DOF accelerometer, and three DOF magnetometer built in, with some redundancy amongst components. Though these sensors are the only ones necessary to fly, we felt that an airspeed sensor would be an asset for our implementation; for hardware-in-the-loop testing through QGroundControl, this sensor was also a necessity. As such, we purchased and integrated one with the flight computer. We felt that the combination of an open source software suite and flight computer allows for maximum customization in this project, especially as we seek to allow for flexibility in the final product (ex. using alternate sensors to satisfy the same goals for PX4).\\

\noindent To test the software on the ground, we are using QGroundControl, run on Ubuntu 17.10 in the lab. The PX4 documentation specifically suggested this application, along with FlightGear, for hardware in the loop testing, since it is free compared to other options. Unfortunately, we struggled to find a way to calibrate the Pixhawk?s sensors and perform testing in QGroundControl, without having to first have the airspeed sensor and radio controller; as a result, we have had to wait until now to begin this full setup process (since both tools have arrived). It is our hope to soon have all components calibrated, off the aircraft, to do some simple simulation of flight and testing of the electronics equipment.\\

\noindent In order to develop useful requirements, we first created a document which employed the EARS requirements capture technique. The document features 40 useful requirements which we hope to test in order to fulfill our other goal of testing these requirements using an existing model checking software.\\

\subsection{Controls}
\noindent The team has dedicated 3 requirement revisions for the controls system. While developing these requirements, the team has been researching other fixed wing UAS to develop an understanding of what works best fundamentally for controls.\\

\noindent The development of the controls system requirements concentrated on safety, functionality, and responsiveness. These requirements include further subsystems and rationale for supporting the requirements. Further information of the controls requirements can be found in the requirements section of this report. \\

\noindent The team has decided that for the first prototype, our UAS will utilize ailerons, flaps, and rear controls. This decision was made to provide control over the aircraft with minimum additional complexity. The tail and wings will have the standard, conventional style of design for our first model. This allows for "simple" calculations to make it easy on building the tail based on the wings and fuselage. It also allows for simple wiring configurations between the controls system and the electronics. \\

\noindent Some considerations the team has regarding future designs of the UAS include double stacked wings, inverted "V", and vertical takeoff and landing. These are much greater in complexity than desired for the team's first design, but are still considered for future prospects. The controls system would change substantially for either case, but is something the team looks forward to developing. \\

\noindent The team is still in the process of researching other projects to determine what controls system designs will be utilized. The Albatross is of particular interest to the team and will likely influence control system and structural decisions next year. The software aspect of the controls is further explained under the software section of this report. \\

\subsection{Electronics}
\noindent After narrowing the overall weight of the UAS to approximately six pounds during the early iterations of the calculated design, the team decided to invest in an Iron Bird model of the UAS for testing. Using the overall weight plus an approximate weight-to-thrust ratio required for minimum takeoff, two motors were selected to fulfill different mission parameters, which are longest flight time and highest attainable velocity. These motors exceeded the minimum required thrust-to-weight ratio so that the UAS may take off in a shorter distance, but also have distinctly different flight parameters for testing the structure and the versatility of the UAS and its customization. Based off of the two motor selections, ESCs, power modules, propellers, batteries and a case to transport the Iron Bird was selected and ordered. Multiple types of propellers were purchased in order in give an even larger envelope for testing the flexibility of the UAS under different conditions. A good example of this is how the propeller characteristics and battery influence flight time, takeoff distance, and max speed. \\

\noindent Due to deciding that the PX4 would be the team's flight computer, and that the team would be using the Airspeed sensor and other needed parts from the purchased Albatross, the team has not ordered additional electronics for now. All parts currently needed for the iron bird version 1 that was ordered separate from the Albatross has been soldered and properly attached to form the first basic testbed system. After a closer look at the system though, an additional set of electronics will be needed to control the servos. We will most likely use a twist-on wire connector to avoid additional soldering and make it easier to debug if a wire becomes loose. \\

\noindent As the project develops, more electronics requirements and rationale are added to demonstrate the growth of the understanding and needs of the electronics in the UAS. \\

\subsection{Propulsion}
\noindent The propulsion system considered to be the propellers and the motor. The battery was considered for this system, but placed specifically in the electronics category because it runs more than just the motor. Propulsion is directly influenced by the electronics, and equally influences the electronics system. This is because for this UAS, the propulsion system was decided to be electronic. This decision was made due to the lightness of the motor system, the fixed and lighter weight of the fuel, and attempting to use up-to-date technology. \\

\noindent When choosing what motors were going to be used, a thrust-to-weight ratio and an overall weight approximation of the UAS was used from the structures side of the team. Although originally planned to be a next semester purchase, a request for the Iron Bird version 1 was requested, so a decisions for electronics and propulsion were pushed forward. The propulsion was prioritized over the electronic decision making because a weight and a thrust-to-weight decision had been made, so finding a motor to fit the needs and basing the battery and other needed parts off of the motor made the most sense. Two motors were decided on, one that had a low draw on the battery, but also had the lowest thrust, and another that had higher draw, weighed more, but also had a significantly higher thrust. This allowed us to have a large range of test parameters and flexibility, with the hope that the team could make the UAS fairly customizable while also keeping it affordable. Both motors were fairly inexpensive, and low weight, with the lighter one sitting at twenty grams, and the other just over eighty grams. \\

\noindent Each motor had to separate sets of propeller purchased for them. They both had a higher thrust set and a lower thrust set, with the idea that the motors could be used for multiple roles. In particular, an idea was made to build a contraption that contained the propeller and motor of the smaller set for live testing within the lab of the Iron Bird. This way, the team could test how the controls responded with the controller and any autonomous programs while under standard, constant power draw. \\

\noindent As more iterations of both the requirements of the structures and the calculated characteristics of the UAS change, the requirements of the propulsion have the potential to change due to values falling outside of the expect initial range. Because of that, many of the valued requirements still have variables in place of the number in order for the team to lock in the desired, but also reasonable, requirements for the UAS. \\

\subsection{Ground System}
\noindent The ground system of this aircraft will include features of takeoff, landing, transmitting and receiving, and an assembly kit. The team is still considering different takeoff options such as a catapult launch or runway launch. These different considerations require their own different requirements such as landing gear or some sort of connection to the catapult. Different landing considerations include bush landing and autonomous runway landing. The team has decided that the kit for setup shall be complete and support an easy assembly and pre-flight check for the UAS. Necessary items for flight include having FAA registration documents, remote pilot certificate, and tools for in-field construction or adjustments. The team will use the following semester to solidify decisions within the ground system as the preliminary design for the UAS develops. \\


%%LESSONS LEARNED%%
\section{Lessons Learned}
\begin{enumerate}
\item Despite best efforts to set timeline at the beginning of the project, we must be prepared to adapt to unexpected changes.
\begin{enumerate}
\item Getting started can be slow as we work to iron out the details
\item Things can arrive broken and must be sent back
\item Some items will arrive but not be what you purchased
\item Things sometimes don't arrive at all
\item Outside parties may not always be capable of acting on our timeline
\end{enumerate}
\item Documentation takes a lot of effort & time, and if not maintained in a ``timely" fashion, can become cumbersome to all.
\begin{enumerate}
\item It's important to act with due diligence to ensure things are up to date, otherwise we fall behind with tracking our progress and can fail to capture the rationale behind our decisions.
\end{enumerate}
\item Purchase orders are not simple within an organization, and many levels of documentation and rationale must be used in order to obtain needed materials.
\begin{enumerate}
\item Rationale requires comparisons to other products of similar quality and attributes.
\item Extensive research may be needed to find the right products
\item A "completed" list of items will always need to be amended
\item Costs must be taken into consideration
\item A "completed" list of items will always need to be amended
\item Multiple item requests are to be expected.
\item Every task has a tool
\end{enumerate}
\item Documentation of safety takes much time, especially when online sources have varied (and untested) solutions.
\begin{enumerate}
\item Double check information found in an area that may not be well research
\item Check sources for validity
\item Observe and note inconsistencies in information
\item Watch and report if an object does something it doesn't normally do
\item Keep track of recommended safety items and keep the lab full of options at all times
\item If sanding/cutting/altering a product, make sure that the particles released in the air are okay for lungs
\end{enumerate}
\item Soldering takes time and patience.
\begin{enumerate}
\item Having extra hands, mechanical or not, makes the process easier
\item Prepare dummy tests at first so that material isn't going to waste
\item Always have spares
\item A hot solder tool does not mean the project will be finished fast/well
\item Always remember heat shrink
\item Don't leave heat shrink too close to the work area
\item Don't breathe in solder fumes
\end{enumerate}
\item Don't expect items to come in on-time or intact. 
\item Expect delays in building parts and putting electronics together.
\begin{enumerate}
\item Troubleshooting skills are needed with software.
\end{enumerate}
\item Daily, respectful reminder emails are appreciated.
\begin{enumerate}
\item Professors, Teacher Assistants, Employees, Mentors, and other individuals helping with projects of this scale have other things to work on. Reminders help keep you in the scope of their day to day work.
\end{enumerate}
\item Hotwire
\begin{enumerate}
\item Cutting a wing out of foam with a hotwire requires a large airfoil.
\item The stoppers on the end of the hotwire need to be taped to keep the wire from falling off.
\item Plastic teplates will melt. Metal templates are required for using the hotwire.
\end{enumerate}
\item 3D Printer
\begin{enumerate}
\item ABS can cause the printing bed to bubble when used for too long. Switch filaments often to avoid bubbling of PEI printing bed.
\item Always perform correct printer bed preparation for the specific filament type.
\item Update CURA often to avoid mid-print malfunctions.
\item After aborting a print job, turn printer off immediately before performing any other actions.
\item LulzBot support center is willing to help if a support ticket is submitted through the LulzBot website.
\end{enumerate}
\end{enumerate}

%%LIST OF FUTURE TASKS%%
\section{Future Tasks \& Deliverables}
\begin{enumerate}
\item Performance analysis of Lithium Polymer batteries versus each other.
\item Performance analysis of Lithium Polymer batteries versus Albatross Lithium Ion trapezoidal battery pack
\item Figure out a way to get the Lithium Ion battery pack to work with our charger
\item Analysis and study of unexpected weight and performance of two of the Lithium Polymer batteries
\item Design multiple fuselages, tails, and airfoils/wings specifically for our UAS
\item Create a launch system
\item Construction of the electronic subsystem within the UAS
\item Complete integration of electronic parts with the control surfaces
\item Further outfit the lab with equipment and computers with needed programs
\item Lab setup completed
\item Final Expanded Battery Safety Documentation
\item Document showing all purchased items for the lab 
\item Document showing successful retrieval of all items 
\item Fully outfitted Iron Bird 
\item Purchase List converted to LaTeX
\item Hardware in the loop testing through QGroundControl and FlightGear. 
\item Flight testing the Albatross
\item Final learned lessons from Albatross contruction
\item Alternative items list purchased for the Albatross
\item Final Design of UAS
\item Flight of UAS
\item Final Documentation of Requirements
\item Final Project Document and Review
\item End of Semester III Paper for Dr. Rozier
\end{enumerate}

%%REFERENCES%%
\unlabeledsection{References}
\begin{thebibliography}{100}
\bibitem{AlbatrossFAQ} Albatross UAV - FAQ. (n.d.). Retrieved December 13, 2017, from \\\url{https://www.appliedaeronautics.com/far/}
\bibitem{Crawley2004} Crawley, E., De Weck, O., Eppinger, S., Magee, C., Moses, J., Seering, W.,Whitney, D. (2004). The Influence of Architecture in Engineering Systems. MIT Engineering Systems Division, 1-29. Retrieved December 11, 2017, from \\\url{http://strategic.mit.edu/docs/architecture-b.pdf}
\bibitem{Terzakis2013} Terzakis, J. (2013). EARS: The Easy Approach to Requirements Syntax. In International Academy, Research, and Industry Association: The Eighth International Multi-Conference on Computing in the Global Information Technology. Retrieved October 11, 2017, from \\\url{https://www.iaria.org/conferences2013/filesICCGI13/ICCGI_2013_Tutorial_Terzakis.pdf}
\bibitem{PX4Git} PX4 Open Source Dev Team. ``PX4." PX4 Pro Autopilot Software, 1.6.5, Github, 11 Dec. 2017, \\\url{github.com/px4/Firmware/}.
\end{thebibliography}


\end{document}
