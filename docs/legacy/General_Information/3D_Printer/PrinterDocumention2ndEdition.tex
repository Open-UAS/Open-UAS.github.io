\documentclass{article}
\usepackage[utf8]{inputenc}

\title{Printer documentation}
\author{troquet }
\date{March 2022}

\begin{document}

\maketitle

\section{Introduction}
The current printer use in the lab is the Lulzbot Taz pro and is apart of the taz line of printers. With the use of this printer we can create new parts for a variety of needs with the use of additive manufacturing. This documentation is meant for the purpose of helping users of this printer to trouble shoot and identify issues that may arise with its use. 

\section{Before anything else}
One important thing to note with the use of the printer is many of the potential problems that comes with 3D printing is in fact not an issue with the printer itself but an issue with the slicer you are using. We use Cura Lulzbot edition as our slicer. Cura is an open source slicer used for a variety of printers. It has many different settings for the use of many different materials and printers. 

\subsection{Cura settings} 
A big issue that arises with the use of the cura is making sure that the software is up to date and that the settings listed in cura are correct for the uses you want. There are many different settings that need to be taken in account as they change depending on the material you are printing. 
Two common types of materials that are used for printing is PLA and ABS. Though they bare a lot of similarities for their uses they have some differences in the needed setting for effective prints. A common difference that results in a problem is the difference in the printing temp. ABS is printed at a higher temperature then PLA and if you are using ABS printing temperature for PLA it will results in a number of issues including but not limited to nozzle clogging, improper bed adhesion, incorrect printed parts. 

\section{Checking Your Model}
Other issues can in fact be caused from the model you are using. When printing you need to make sure that any holes and or connection pieces are oversized. We recommend using .01 to .02 inches depending on the use of the part. 

\section{Calibration}
A less common issue with the TAZ Pro is the need for calibration. There are few Types of calibration that 3D printers may need. This includes flow rate and Axes calibration. In reality this is most likely not an issue unless prints are having issues with skewking or    
\end{document}
