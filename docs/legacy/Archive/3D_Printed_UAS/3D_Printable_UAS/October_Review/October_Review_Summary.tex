\documentclass[]{article}
\usepackage[margin=1in]{geometry}
\usepackage[affil-it]{authblk}
\usepackage{gastex} % pictures on steroids
\usepackage{epsfig}
\usepackage{float}
\usepackage{graphicx}
\usepackage{caption}
\usepackage{subcaption}
\usepackage{amsmath}
%\usepackage{times}
%\usepackage{mathtime} 
\usepackage{pslatex} %regular times font/smaller courier font
\usepackage{amssymb}
\usepackage{alltt}
\usepackage{longtable}
\usepackage{multirow}
\usepackage{url}
\usepackage{mathrsfs} %for \mathscr{} 
\usepackage{multicol}
\usepackage{capt-of}
\usepackage[toc,page]{appendix}
\usepackage{booktabs}
\usepackage{listings}
\usepackage{color}
\usepackage{nomencl}
\usepackage{hyperref}


\begin{document}

\section{Summary}
\subsection{"The Seven Intellectual Pivot Points for Conceptual Design"}
\begin{itemize}
\item {\bf Requirements} This is where you establish the set of requirements the aircraft must meet.  

\item {\bf Weight of the Airplane - First Estimate} Getting an initial estimate of the airplane's weight is important for determining the performance parameters.

\item {\bf Critical Performance Parameters}  This is where the airfoil will be chosen to determine some initial aerodynamic characteristics.

\item {\bf Configuration Layout} Here we will determine more specifics about the design of our aircraft. 

\item {\bf Better Weight Estimate} Based on the configuration and initial prototype, a better weight estimate can be determined.  

\item {\bf Performance Analysis} Are we meeting or exceeding our requirements?  If not, iterate steps until we converge on a solution.

\item {\bf Optimization} Is this the best design?
\end{itemize}

\subsection{Requirements}
\begin{itemize}
\item Easily re-configurable to accommodate different sensor suites and adequate internal cooling fans
\item Long battery life to enable distance missions where autonomy is required
\item Small size that enables flight testing at a variety of locations
\item Easily decomposable for easy transportation
\item 3D printable parts for quick prototyping 
\end{itemize}

\subsection{Trade Studies}
\begin{itemize}
\item Previously Successful Small UAS
	\begin{itemize}
	\item The Pointer, Raven, Puma, Dragon Eye, Desert Hawk, Orbiter, Razor
	\end{itemize}
\item High wing vs low wing configuration
\item Tractor motor vs pusher motor configuration
\end{itemize} 

\subsection{Weight of the Airplane - First Estimate}
\begin{itemize}
\item MATLAB/Python code
	\begin{itemize}
	\item Determines initial guess of take-off weight, given a certain payload	
	\item Produces graphs of some initial performance parameters
	\item Defines constraint space for this aircraft using the following equations:
	\subsubsection{Constraint Equations}:
	
	Max Load/Turn:
	\begin{equation}
	\frac{HP}{W}= \frac{1}{550\eta_{p}}[\frac{1}{2}\rho V^{3} C_{D_{O}} (\frac{S}{W}) + 2K \frac{n^2}{\rho V} (\frac{W}{S})]
	\end{equation}
	
	Endurance:
	\begin{equation}
	\frac{HP}{W}=\frac{4}{550\eta_{p}} C_{D_{O}}^{1/4} (\frac{K}{3})^{3/4} (\frac{2}{\rho} \frac{W}{S})^{1/2}
	\end{equation}
	
	Cruise:
	\begin{equation}
	\frac{HP}{W}=\frac{2}{550\eta_{p}} C_{D_{O}}^{1/4} K^{3/4} (\frac{2}{\rho} \frac{W}{S})^{1/2}
	\end{equation}
	
	Takeoff Distance:
	\begin{equation}
	\frac{HP}{W}=\frac{2.44}{550\eta_{p}} \frac{1}{gd_{to}}(\frac{1}{\rho_{SL} C_{L_{max}}} \frac{W}{S})^{3/2}
	\end{equation}
	
	Stall Condition
	\begin{equation}
	\frac{W}{S}= \frac{\rho}{2} C_{L{max}} V_{SO}^2
	\end{equation}
	
	
	\end{itemize}	
\end{itemize}

\subsection{Configuration Layout}

\begin{itemize}
\item 3D Modeling using open source FreeCAD
\end{itemize}

\end{document}