\documentclass{article}
\usepackage[margin=1in]{geometry}
\usepackage[affil-it]{authblk}


%
%This is a LaTeX document for organizing the Related Work on the 3-D Printed UAS project
%
% To compile this into a .pdf simply run 'pdflatex 3DUAS_RelatedWork.tex' on the command line.
% To automatically generate the bibliography, then run 'bibtex 3DUAS_RelatedWork' and then repeat the command 'pdflatex 3DUAS_RelatedWork.tex' twice.
%

\usepackage{amsmath}
%\usepackage{times}
%\usepackage{mathtime} 
\usepackage{pslatex} %regular times font/smaller courier font
\usepackage{amssymb}
\usepackage{alltt}
\usepackage{longtable}
\usepackage{multirow}
\usepackage{url}
\usepackage{mathrsfs} %for \mathscr{} 
\usepackage{multicol}
\usepackage{capt-of}
\usepackage[toc,page]{appendix}
\usepackage{booktabs}
\usepackage{listings}
\usepackage{color}

\lstset{frame=tb,
  aboveskip=3mm,
  belowskip=3mm,
  showstringspaces=false,
  columns=flexible,
  basicstyle={\small\ttfamily},
  numbers=none,
  numberstyle=\tiny\color{gray},
  keywordstyle=\color{blue},
  commentstyle=\color{dkgreen},
  stringstyle=\color{mauve},
  breaklines=true,
  breakatwhitespace=true,
  tabsize=3
}

%\newtheorem{thm}{Theorem}[section]
\newtheorem{cor}{Corollary}
\newtheorem{lem}{Lemma}
\newtheorem{defin}{Definition}
%Use like this: \begin{lem}\label{...}  .... \end{lem}

%%%%%%%%%%%%%%%%%%%%%%%%%%%%%%%%%%%%%%%%%%%%%%%%%%%%%%%%
% Figure Magic
%%%%%%%%%%%%%%%%%%%%%%%%%%%%%%%%%%%%%%%%%%%%%%%%%%%%%%%%
\usepackage{gastex} % pictures on steroids
\usepackage{epsfig}
\usepackage{float}
\usepackage{graphicx}
\usepackage{caption}
\usepackage{subcaption}
\renewcommand{\topfraction}{.95} %figures can take up at most 95% of the page before being alone
\renewcommand{\bottomfraction}{.99} %figures can take up at most 99% of the page before being alone
\renewcommand{\textfraction}{.1} %at most this this % of page will be text before making figure-only page

%%%%%%%%%%%
% Block Quotes       %
%%%%%%%%%%%
\newenvironment{blockquote}{
  \par
  \medskip
  \leftskip=4em\rightskip=2em
  \noindent\ignorespaces}{
  \par\medskip}

%for code listings
%\usepackage{listings}
%\usepackage{courier}


\title{3-D Printed Unmanned Aerial System Project}
\author{
Jessica Glass %\inst{1}
and Kristin Yvonne Rozier %\inst{1}
}
\affil{University of Cincinnati, Cincinnati, Ohio, USA \\
   \texttt{\{glassjp,rozierky\}@ucmail.uc.edu}
}
\date{\today}


\begin{document}

\maketitle

%
% Add a table of contents to make the document easier to index as it grows...
%
\tableofcontents


\section{Project Overview}
The Laboratory for Temporal Logic (LTL) in Aerospace at the University of Cincinnati is designing a new, free-and-open-source, easy-manufacture (i.e. 3-D printable with COTS components) Unmanned Aerial System (UAS). The UAS will be designed specifically to maximally support System Health Management (SHM) capabilities needed to safely create intelligent, autonomous UAS.

\section{Initial Design Estimates}
This section is a collaboration of tables and graphs which illustrate some initial 2-D flight characteristics for different airfoils.  This information will be used in order to make a decision for which airfoil will be used on the 3-D printable UAS.  This 3-D printable UAS will fly at relatively low Reynold's numbers.  The analysis in this section is performed using the open-source software XFLR5.  All airfoils have been analyzed for a range of Reynold's numbers, as well as a range of angle of attack.  The Reynold's numbers for this analysis have been ranged from 2,000 to 333,000, based on the historical flight regime for typical small UASs.  This gives a range of speeds from 15 m/s to 25 m/s. The angle of attack has been ranged from 0 degrees to 15 degrees which is a reasonable range of AOA for this flight regime and allows us to visualize the stall AOA for all Reynold's numbers.

\subsection{SD7003}
This airfoil is a commonly used glider airfoil. Max thickness location is further forward than typical glider airfoils.






\end{document}