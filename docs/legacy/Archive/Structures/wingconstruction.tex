\documentclass{article}
\usepackage[margin=1in]{geometry}

\begin{document}

\title{\textbf{Wing Construction Outline}}
\author{Abigail Gries}
\maketitle

This is a step-by-step guide explaining how to construct the UAS wings out of foam. \\


\section*{Materials/Resources}

\begin{itemize}
	\item Large hot wire capable of reaching at least 12 volts
	\item Exacto knife
	\item Heavy weight
	\item EPP Foam (at least 6 ft x 1 ft x 2 in)
	\item Masking or duct tape
	\item Wood or metal (for airfoil template)
	\item Table or surface to work on 
	\item Nearby outlet
	\item Box fans
	\item Two 1/8th inch screws
	\item Tape measure
	\item Sharpie or pen

	
\end{itemize}

\section*{Procedure}
\begin{enumerate}
	\item Create a wooden or metal template of the NACA 4512 airfoil. Make sure the material can withstand high heats and has screw two screw holes.
	\item Measure out 2 ft 7 inches on the foam. Mark with a line across the foam.
	\item Measure out 6 inches on the foam (width). Mark with a line across the foam.
	\item Set up hotwire according to instructions. Make sure that the cord can reach entire distance while cutting. 
	\item Set up box fans and open doors so that fumes will exit the area while cutting.
	\item Lay foam on cutting surface and place a heavy weight on the top of the foam.
	\item Turn hotwire on to 12 volts.
	\item Cut foam long ways first - across the 2 ft 7 in line. The key is to go very slowly without stopping. Do not apply much pressure - let the hot wire do 					work for you.
	\item Turn hot wire off
	\item Let wire cool before wiping any foam residue off with paper towel

\end{enumerate}


\end{document}

