\documentclass{article}
%%%%%%%%%%%%%%%%%%%%%%%%%%%%%%%%%%%%%%%%%%%%%%%%%%%%%%%%%%%%%
% Lecture Specific Information to Fill Out
%%%%%%%%%%%%%%%%%%%%%%%%%%%%%%%%%%%%%%%%%%%%%%%%%%%%%%%%%%%%%
\newcommand{\LectureTitle}{NEAR Lab Discussion Notes}
%\newcommand{\LectureDate}{\today}
\newcommand{\LectureDate}{April\ 6,\ 2016}
\newcommand{\LatexerName}{LTL}
%%%%%%%%%%%%%%%%%%%%%%%%%%%%%%%%%%%%%%%%%%%%%%%%%%%%%%%%%%%%%

\usepackage{amsmath,amsfonts,amsthm,amssymb}
\usepackage{setspace}
\usepackage{Tabbing}
\usepackage{fancyhdr}
\usepackage{lastpage}
\usepackage{extramarks}
\usepackage{chngpage}
\usepackage{soul,color}
\usepackage{graphicx,float,wrapfig}
\usepackage{afterpage}
\usepackage{abstract}

\topmargin=-0.45in
\evensidemargin=0in
\oddsidemargin=0in
\textwidth=6.5in
\textheight=9.0in
\headsep=0.25in

\pagestyle{fancy}
\lhead{\LatexerName}
\chead{\LectureTitle}
\rhead{\LectureDate}
\lfoot{\lastxmark}
\cfoot{}
\rfoot{Page\ \thepage\ of\ \pageref{LastPage}}
\renewcommand\headrulewidth{0.4pt}
\renewcommand\footrulewidth{0.4pt}


\begin{document}
\begin{spacing}{1.1}
\newpage

\begin{itemize}

\item Materials/3D printing suggestions

\begin{itemize}
\item PLA - good material - brittle, stiff, lightweight, easy to print
\item Some PLA manufacturers have porr quality control resulting in irregular filament diameter -- this affects extrusion rate
\item Orientation does matter -- materials will snap between layers when load is applied
\item Polycarbonate has very strong layer bonding but it is difficult to print
\item Nylon is extremely strong but also flexible --good for parts where stiffness doesn't matter
\item  McMaster-Carr can supply many COTS items to reduce number of printed parts (they have everything, an engineer's best friend)
\item SOME 3D PRINTING COMPANIES NAMED (unsure of spelling on all)
\begin{itemize}
\item RepRap - Great community, open source, good for tinkering and experimenting
\item Makers Tool Works - Mendel Max
\item SeeMe CNC - orion delta printers -- good at printing circles
\item Lulz Bot ** This is the printer he really raved about. Insisted that you pay extra but for really great quality -- best printer on the market in his opinion
\item DO NOT USE MAKER BOT- nozzle breaks often and replacing parts is difficult
\end{itemize}
\end{itemize}

\item Control Surfaces
\begin{itemize}
\item prefers direct drive
\item has flown aircraft with little to no control surfaces
\item will need to come up with mechanism to wire motor to battery if we use rudders
\end{itemize}

\item Battery
\begin{itemize}
\item Run voltage as high as possible --allows most efficient way to power aircraft
\item Use power regulators to regulate power to components
\item One power regulator mentioned was BECS -- very useful because it is reprogrammable

\end{itemize}


\item Wing Design

\begin{itemize}
\item suggested that wing should not be 3D printed
\item One idea: Carbon fiber spar, 3d printable rib, fabric covering (talk to AeroCats)
\item Another idea: Foam wing --carbon fiber or fiberglass as needed
\end{itemize}

\item Motor/Props
\begin{itemize}
\item Out runner motor
\item With our style of aircraft we should use big motor, big prop (overdo it if anything)
\item BIG speed controller (well above rated amp)
\item Castle speed controllers - easily reprogrammable
\item RC groups will have motor recommendations
\item Some Brand Names mentioned:
\begin{itemize}
\item Hacker if possible ** may have gone out of business
\item Neu Motor was another very good company  named
\item Axi
\item Hyperion
\end{itemize}
\item Props:
\begin{itemize}
\item Fixed Prop
\item Wooden props work well and are very light weight
\item Composite prop will last longer - carbon fill carbon resin
\item Some Companies Named: Graupner, Aeronaut, Zinger, Top Flight
\end{itemize}
\end{itemize}


\item Miscellaneous 

\begin{itemize}
\item Do not rule out hand-launch -- as long as the aircrat is statically stable it should be able to be let go and fly on its own
\item the design should be driven by the payload -- battery should be very last consideration.
\item For hand launch, low pitch prop
\item Minimize vibration in general but especially if using camera
\item Camera payload considerations
\begin{itemize}
\item Global shutter - captures all pixels at once
\item Will need to know when to take pictures
\end{itemize}
\item Payload does not include anything needed to fly aircraft, so all motors should be taken out of payload calculation
\end{itemize}



\end{itemize}

\end{spacing}
\end{document}
