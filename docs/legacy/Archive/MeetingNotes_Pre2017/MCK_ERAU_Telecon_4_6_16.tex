\documentclass{article}
%%%%%%%%%%%%%%%%%%%%%%%%%%%%%%%%%%%%%%%%%%%%%%%%%%%%%%%%%%%%%
% Lecture Specific Information to Fill Out
%%%%%%%%%%%%%%%%%%%%%%%%%%%%%%%%%%%%%%%%%%%%%%%%%%%%%%%%%%%%%
\newcommand{\LectureTitle}{McKenzie ERAU Teleconference Notes}
%\newcommand{\LectureDate}{\today}
\newcommand{\LectureDate}{April\ 6,\ 2016}
\newcommand{\LatexerName}{LTL}
%%%%%%%%%%%%%%%%%%%%%%%%%%%%%%%%%%%%%%%%%%%%%%%%%%%%%%%%%%%%%

\usepackage{amsmath,amsfonts,amsthm,amssymb}
\usepackage{setspace}
\usepackage{Tabbing}
\usepackage{fancyhdr}
\usepackage{lastpage}
\usepackage{extramarks}
\usepackage{chngpage}
\usepackage{soul,color}
\usepackage{graphicx,float,wrapfig}
\usepackage{afterpage}
\usepackage{abstract}

\topmargin=-0.45in
\evensidemargin=0in
\oddsidemargin=0in
\textwidth=6.5in
\textheight=9.0in
\headsep=0.25in

\pagestyle{fancy}
\lhead{\LatexerName}
\chead{\LectureTitle}
\rhead{\LectureDate}
\lfoot{\lastxmark}
\cfoot{}
\rfoot{Page\ \thepage\ of\ \pageref{LastPage}}
\renewcommand\headrulewidth{0.4pt}
\renewcommand\footrulewidth{0.4pt}


\begin{document}
\begin{spacing}{1.1}
\newpage

\begin{itemize}
	\item 3D Printers
	\begin{itemize}
		\item DO NOT USE MAKERBOT
		\begin{itemize}
			\item nozzle breaks often, replacing parts is a hassle
		\end{itemize}
		\item RepRap
		\begin{itemize}
			\item Great community, open source, good for tinkering
		\end{itemize}
		\item Makers Toolworks Mendel Max
		\item SeeMe CNC Orion Delta
		\item LulzBot
		\begin{itemize}
			\item Best bang for your buck, one of the best printers on the market
		\end{itemize}
		\item Ultimaker
	\end{itemize}
	\item Printing Materials
	\begin{itemize}
		\item PLA
		\begin{itemize}
			\item Depends on manufacturer; poor quality control can lead to variable filament diameter, which affects extrusion rate
			\item Stiff, lightweight, easy to print
			\item Will snap, especially in between layers
		\end{itemize}
		\item Polycarbonate
		\begin{itemize}
			\item Will not break, layers essentially disappear into each other, but difficult to print
		\end{itemize}
		\item Nylon
		\begin{itemize}
			\item Very bendy, super strong
		\end{itemize}
		\item Layer adhesion does pose a challenge, as it makes printed materials weak. Orienting a part so that forces acting on it in the air will not cause failure in between layers is important
	\end{itemize}
	\item Battery
	\begin{itemize}
		\item Use highest voltage possible - pick motor that works with it
		\item Choose after complete payload design
		\item Just need regulators and speed controllers
		\begin{itemize}
			\item Castle Speed Controllers - Castle Link can be used to reprogram anywhere anytime
			\item BECS
		\end{itemize}
	\end{itemize}
	\item Motor
	\begin{itemize}
		\item Overbuild, overprop
		\item Big speed controller
		\item Motors to consider: NEU, Hacker, Axi, Hyperion
		\item Go with an outrunner, don't make it a pusher
		\item RC groups will have motor recommendations
	\end{itemize}
	\item Props
	\begin{itemize}
		\item Fixed props
		\item Wood is good, but composites last longer
		\item carbon fill carbon resin
		\item Props to consider: Graupner, Aeronaut, Zinger
	\end{itemize}
	\item Design Ideas
	\begin{itemize}
		\item 3D print ribs and cover wings with fabric
		\begin{itemize}
			\item Notes from Aerocats
			\begin{itemize}
				\item Balsa, basswood, used for frame and spars
				\item Wings are covered in MonoKote or UltraCote which is essentially a heat shrink heavy duty Saran wrap
				\item MonoKote - adhesive on one side, manufactured by Top Flite
				\item UltraCote - same as MonoKote, manufactured by Horizon
				\item These wraps seem easy to apply and very affordable, according to Mark Fellows they are both incredibly strong, as the Aerocats planes carry ~30lb payloads
			\end{itemize}
		\end{itemize}
		\item Foam wings with carbon fibber spars
		\begin{itemize}
			\item Print airfoil guide for hot wire
			\item Make wings modular, able to use different wings
		\end{itemize}
		\item Camera payload
		\begin{itemize}
			\item Global shutter - All pixels at once
			\item Rolling shutter - one line at a time, not good with camera movement
			\item Determine when to take pictures, minimize vibrations
		\end{itemize}
		\item Hand launch
		\begin{itemize}
			\item Statically stable aircraft
		\end{itemize}
	\end{itemize}
\end{itemize}

\end{spacing}
\end{document}
