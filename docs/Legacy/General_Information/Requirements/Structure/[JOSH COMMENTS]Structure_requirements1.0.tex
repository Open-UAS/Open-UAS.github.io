\documentclass{article}
\usepackage[margin=1in]{geometry}
\usepackage{color}

\begin{document}
	\title{Structure Requirements (Rough Draft)}
	
	\author{Catherine Sener, Abigail Gries}
	\maketitle
	
	\textit{This document will describe the structures requirements for the Iowa State University Open UAS project.}\\
	
	\flushleft{\textcolor{red}{Overall, I think justifications should probably be added to these. Not because they aren't good requirements,
	but because that satisfies one of our goals of documenting and justifying our decisions.}}
	
	\begin{enumerate}
		
		\section{Wings}
		
		\item The wings shall be able to withstand drafts of up to 50 knots.\\
		\item The wings shall be constructed of EPP foam.\\
		\item The wingspan shall be 5 feet.\\
		\item The wing planform \textcolor{red}{(typo?)} area shall be 2 square feet.\\
		\item The maximum coefficient of lift shall be upwards of 1.8.\\
		\item The wing loading factor shall not exceed 3.5 pounds per square foot.\\
		\item The wings shall be attached to the frame of UAS by a 3D printed latch. \textcolor{red}{Do we want to assert this yet, or 				should it be left more ambiguous? - Like, "the wings shall be removable" instead}\\
		\item The wings shall be removable through the printed latch.\\
		\item The wings shall contain a rod through the center to add weight.\\
		\item The wings shall have control surfaces.\\
		\item The wings shall have multiple paths inside to allow wires for control surfaces.\\
		\item The foam wings shall be covered by a balsa wood shell.\\
		
	\end{enumerate}
		
	\begin{enumerate}
		
		\section{Fuselage}
		
		\item The fuselage shall house the required electronic equipment.\\
		\item The inside of fuselage shall cushion and protect components.\\
		\item The fuselage shall contain custom 3D printed storage containers for components.\\
		\item The inside of fuselage shall be accessible.\\
		\item The fuselage shall be able to withstand minimum impact of 2G.\\
		\item The fuselage shall be aerodynamically efficient (??).\textcolor{red}{At this stage, we may want to quantify this}\\
	
	\end{enumerate}

	\begin{enumerate}
		
		\section{Empennage}
		
		\item Empennage shall have appropriate control surfaces (??). \textcolor{red}{At this stage, we may want to quantify this}\\
		
	\end{enumerate}

	\begin{enumerate}
		
		\section{Materials}
		
		\item The 3D printed components shall be produced by the LulzBot Taz 6 3D printer.\\
		\item The 3D printed structural components shall use ABS filament.\\
		\item The non-structural 3D printed components shall use PLA filament.\\
		\item The foam used for the wings shall be EPP foam.\\
		\item The foam shall only be cut using the hot wire and hot knife.\textcolor{red}{I'm not sure if this should be captured as a 			Materials requirement, or instead elsewhere in safety documentation - doesn't really describe the material itself}\\
		\item The materials shall be accessible by general public.\textcolor{red}{For clarity, perhaps explain that the general public should 		be able to purchase them, rather than just 'access' them}\\
		\item Alternate materials and printers shall be listed for hobbyists' use.\textcolor{red}{(I'm not sure if this should be captured as a 			Materials requirement, or instead appear in some other documentation - it implicitly requires that the design be such that 				here are multiple possible materials, but describes the method instead)}\\

	\end{enumerate}		

	\begin{enumerate}
		
		\section{General}
		
		\item All components shall be placed such that the center of gravity is stable.\\
		\item All connection points shall be able to withstand minimum impact of 2G.\\
		\item All components shall be modeled and documented in SolidWorks. \textcolor{red}{(Seems like a description of the project itself, or perhaps the process, but not of the actual subsystem in question)}\\
		\item The weight of the total structural frame without electrical components shall not exceed 2.5 lbs.\\
		\item The entire frame shall be waterproof and protect housed electrical components from water.\\
		
		
		
	\end{enumerate}
	
\end{document}
