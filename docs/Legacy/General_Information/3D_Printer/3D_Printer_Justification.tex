\documentclass{article}
\usepackage[margin=1in]{geometry}

\begin{document}

\title{\textbf{3D Printer Selection}}
\author{Abigail Gries}
\maketitle

\section*{Part 1: Top Printer Choices}
There were five main 3D printers that the OpenUAS team researched. This section describes the main features of each printer. These printers were selected as top candidates due to reviews from popular 3D printing websites.

\subsection*{MakerBot Replicator+}
The MakerBot Replicator+ is a large desktop printer sold by MakerBot. The unit costs 2500, and has a print volume of 563 cubic inches. This printer is optimized for PLA filament, a cheap, strong plastic. Additionally, features offered with this product are an onboard camera, connectivity through wifi, usb, and ethernet, and a LCD display. The printer uses MakerBot Print, a free software for Mac and Windows.

\subsection*{LulzBot Taz 6}
The LulzBot Taz 6 is a desktop printer sold by LulzBot for 2500. The Taz 6 has a print volume of 1,185 cubic inches and prints at an average speed of 30-50 mm per second. It supports many different filament types such as ABS, PLA, HIPS, PVA, and many more. The Taz 6 is compatible with various different software programs for 3D printing. Some additional features offered with this product are self-leveling and self-cleaning capabilities.

\subsection*{LulzBot Mini}
The LulzBot Mini is a small desktop printer also sold by LulzBot. This printer retails for 1250. The print volume is 223 cubic inches, and is compatible with the most common 3D printer filaments. Like the Taz 6, the Lulzbot Mini is compatible with various different software programs for 3D printing. This product is advertised as great for beginners.

\subsection*{Fortus 400mc}
The Fortus 400mc is a large, maufacturing 3D printer. The print volume is 1400 cubic inches. The price of this printer is 185,000. The reason this printer was researched was because the University of Virgina built their 3D printed UAS using this printer. They cited a specific filament only used by this printer, Ultem 9085. This filament is commonly used for aerospace designs, due to its strength and light weight. This is an extremely high quality printer, often used by businesses. 

\subsection*{Ultimaker2+}
This printer is sold by Ultimaker. It costs 2500, and it has a build area of 622 cubic inches. Many different materials such as PLA, ABS, CPE, etc can be used with the printer. The printer is open source, uses Cura software, has a build speed of up to 24 mm cubed per second. The nozzles can be easily changed for differetn printing needs. 

\section*{Part 2: Selection}
The team decided to purchase the LulzBot Taz 6 as the lab's 3D printer. There were many reasons behind this decision and factors that ruled out the other printers. The Fortus 400mc was clearly the best printer researched. However, this printer was out of the team's budget and out of regular hobbyists' budget, too. Due to this, the Fortus 400mc was eliminated. The LulzBot Mini was also eliminated because the print volume was not large enough. The team is looking to build larger components with the printer, and the LulzBot website stated that this printer is best for beginners. The MakerBot Replicator+ was taken out of the running by advice from Professor Rozier. She spoke to faculty at Embry Riddle who told her that the 3D printers from MakerBot often were unreliable and produced uneven prints. This left the Ultimaker2+ and the LulzBot Taz 6 to choose from. The LulzBot Taz 6 was chosen because it has a larger print volume for the same price, was recommended by Embry Riddle faculty, one of our members has already worked with it, and there are many positive reviews on the internet for it. 

\end{document}
