\documentclass[letterpaper,11pt]{article}

\usepackage[top=1in, bottom=1in, left=1in, right=1in]{geometry}
\usepackage{listings,verbatim,alltt,amsfonts,times, url}
%\usepackage{url, graphics, graphicx, multirow, amsmath, subfigure, pslatex, listings, verbatim, amssymb, alltt, algorithmic, multirow, graphicx, amsfonts, wrapfig, boxedminipage, rotating, times}
%\usepackage[vlined,linesnumbered,ruled,boxed]{algorithm2e}
%\usepackage[usenames]{color}

%\newcommand{\name}{ACIRS}

% \title{ \textbf{Automated and Adaptive Security for the Enterprise Cloud}}

% \author{}
% \date{}

\begin{document}
%  \maketitle

%{\noindent \textbf{DATA MANAGEMENT PLAN}}
\section*{Facilities, Equipment, and Other Resources}

\subsection*{Student Resources for Education: Open 3D-Printable/COTS UAS Design with Teaching Modules}

There are four 3D printers with a range of different printing capabilities located in the main department office and two more in the extensive additive manufacturing laboratory at the Rapid Prototyping Center, along with milling machines and other equipment to support an academic UAS implementation.\footnote{daapspace.daap.uc.edu} This extensive facility also includes such additive manufacturing equipment as a Concept Laser Mlab laser powder-bed fusion (LPBF) system capable of production of small parts in various alloys including Ti64, Inconels, and cobalt chromium; a EOS M250 LPBF system capable of production of small parts in a bronze material (Direct Metal 20); a Markforged Mark One capable of producing parts in nylon or nylon reinforced with continuous fiberglas, kevlar, or carbon fiber; a Rostock V2 Max capable of printing parts in polylactic acid (PLA) and acrylonitrile butadiene styrene (ABS) thermoplastics; a full suite of support and post-processing equipment such as a Mahr-Federal MahrSurf PS1 surface profilometer used to measure surface roughness during material development efforts. All of this equipment will be available for use of the students working on this educational project. 

In addition to the facilities and DAAP lab staff, the Department of Aerospace Engineering and Engineering Mechanics has a dedicated Senior Research Associate and Adjunct Professor on staff whose duties include design and support of laboratory facilities and experiments. He will actively support the undergraduate student accessing and setting up the equipment used to design an Open-Source, 3D printable Academic UAS Design. He has supported many other faculty members and their students in 3D manufacture projects and will ensure students have the resources in place to effectively accomplish the UAS design project.



\end{document}
